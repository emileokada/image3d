\documentclass[a4paper]{article}

\def\npart {SUROP}
\def\nterm {Summer 2016}
\def\ncourse {Shadow tomography}

\input{header}

\begin{document}
\title{
    SUROP \\ Progress report
}
\title{Summer Undergraduate Research Opportunities}
\date{July, 2016}
\author{Emile Okada \\ University of Cambridge}
\maketitle

\newpage

\setcounter{section}{0}
\section{Week 1}
\subsection{Reading}
I started the week reading Chapter 1 section 2 of Charles L. Epstein's "Introduction to the Mathematics of Medical Imaging".
It covered how to reconstuct a 2d convex object from the shadows of an object. 
The idea is fairly straighforward. 
If $h(\theta)$ is the shadow function as described in the book (essentially the distance of the support line in direction $(-\sin(\theta),\cos(\theta))$ from the origin), then the convex hull can be parameterized by
\begin{equation}
    (x(\theta),y(\theta)) = h(\theta)\cdot(\cos(\theta),\sin(\theta))+h'(\theta)\cdot(-\sin(\theta),\cos(\theta)).
\end{equation}
This idea can be extended to 3d by considering slices of the object. 
Fix some vector {\bfseries v} and then consider the collection of planes perpendicular to {\bfseries v}. 
In each of the planes one can use the 2d method to construct a 2d convex hull of the intersection of the object with the plane. 
Stringing all these 2d slices together then gives a rough reconstruction of the 3d object from its shadows.

I also spent some time reading up on the TV transform and scale spaces, to get a rough idea of which project I'd like to do. 
I ended up going with the tomography project, but spend roughly 1.5 days doing reading for the other project.
\subsection{Coding}
On Thursday I started coding. I've implemented the above idea in python with the following preliminary results.
\subsection{Building}

\section{Code}
\subsection{Euler's Method}
\label{alg:euler}
\lstinputlisting[language=Matlab]{../shadow_function.py}
\end{document}

